\chapter{Generalidades}
\section{INTRODUCCIÓN}
Las tecnologías de información y las comunicaciones proporcionan herramientas eficaces para dar solución a diversos problemas generados por el estilo de vida moderno. En este caso particular se brindará una herramienta que fomentará los buenos hábitos alimenticios en la niñez y ofrecerá a los padres de familia opciones prácticas, nutritivas y balanceadas que permitan el correcto desarrollo y crecimiento de los niños.\\
Desde el enfoque de la ingeniería de software, se propone una solución que aprovecha la portabilidad y el fácil acceso de las tecnologías móviles dado por el estilo de vida moderno para llegar a un sector más amplio de la población.  Por lo anterior se plantea el diseño y prototipado de una aplicación móvil, empleando el lenguaje DART y framework Flutter para su desarrollo, compuesta por varios módulos que le permitirán al usuario consultar opciones de menú e ideas de meriendas balanceados y que aporten los micro-nutrientes que se requieren para el correcto desarrollo físico y mental de niños en el rango de edad de 2 a 10 años.\\
\section{TITULO Y DEFINICIÓN DEL TEMA DE INVESTIGACIÓN}
\subsection{Título}
Prototipo de Aplicación Móvil Para La Nutrición Balanceada de Niños entre 3 a 8 Años de Edad.\\
\subsection{Definición del Tema de Investigación}
La malnutrición infantil es una problemática que genera un alto impacto social y económico a corto y a largo plazo y que en Colombia se ha convertido en un problema de salud pública, pues según cifras de Instituto Nacional de Salud [1] el 25\% de los niños menores de 5 años padece de Anemia.\\
Para generar sociedades más competitivas, capaces de afrontar los desafíos de un mundo cada vez más globalizado se requieren ciudadanos sanos y con pleno desarrollo de sus capacidades físicas y mentales. \\\\
La Investigación podría traer como resultado que la problemática no está directamente relacionada con la capacidad adquisitiva de las familias, si no dado por los cambios en la forma en que se alimenta la sociedad actual, la gran cantidad de comida procesada disponible, la falta de tiempo para preparar los alimentos, pero sobre todo el desconocimiento total de lo que se esta consumiendo.

\section{ESTUDIO DEL PROBLEMA DE INVESTIGACIÓN}
\subsection{Planteamiento del problema}
La malnutrición Infantil por exceso es una problemática que se viene presentando en familias de diferentes niveles socio económicos y no necesariamente en familias con mayores ingresos.\\

Particularmente en Colombia el problema puede derivarse de un factor cultural, asociado a nuestra comida típica y agravado la alta cantidad de azúcar y grasas encontrados en los alimentos procesados que se encuentran en los supermercados sumado al sedentarismo y falta de actividad física característica de la sociedad actual.\\
En la actualidad se observa un alto índice de obesidad infantil producto de la malnutrición por exceso, debido a la cantidad excesiva de carbohidratos y azucares que traen los alimentos algunos prefabricados que nuestra sociedad emplea como alimento ya sea en la lonchera de los niños o en el diario vivir de las familias.\\\\

Estos alimentos no cumplen con la cantidad regulada de azucares además de carecer de las vitaminas, minerales y demás ácidos grasos necesarios para el correcto desarrollo y crecimiento de la población infantil.\\\\
En ciudades como Medellin Colombia, el “48,9\% de los jóvenes agrega sal a las comidas, aumentando el riesgo de hipertensión. En esta ciudad se supera el promedio nacional en consumo de fritos; el 20\% consume comidas rápidas más de tres veces por semana y el 86,4\% consume alimentos de paquete regularmente”. [1]\\

Por otro lado tener abundancia en alimentos no garantiza el acceso a ellos; “aunque los alimentos estén disponibles no se distribuyen equitativamente; el ciudadano medio de los países desarrollados consume 50\% más calorías y 70 \% más proteínas que el habitante promedio de los países pobres. En realidad, la cuestión alimentaria mundial encubre disímiles situaciones nacionales, que no se relacionan con la disponibilidad global de alimentos sino con su distribución y acceso.”[2] 
\\\\
Se propone el desarrollo de un prototipo de una aplicación móvil con el fin de aprovechar la alta demanda y consumo de las tecnologías de información para llegar a más personas, con el propósito de generar conocimiento sobre alimentación saludable en los padres de familia  y ayudar a cambiar cualquier mis concepción sobre el tema que pudiera haber. 
\subsection{Formulación del problema}
¿Qué se puede hacer desde las tecnologías de información para ayudar a los padres de familia ha cambiar hábitos alimenticios que propician el desarrollo de las llamadas "Enfermedades del siglo XXI", en la población infantil en el rango de edad de 3 a 7 años en el colegio Gimnasio Nuevo Despertar de la localidad de Fontibón en Bogotá?
\subsection{Sistematización del problema}
\begin{itemize}
	\item ¿Cuales serían las herramientas de investigación que aplicadas, permitan obtener los datos que se requieren para el estudio?
	\item ¿Qué factores están generando malnutrición por exceso en los niños de 3 a 7 años?
	\item ¿Cómo puede una aplicación móvil ayudar a mejora los hábitos alimenticios de la población estudiada?
\end{itemize}
\newpage
\section{OBJETIVOS DE LA INVESTIGACIÓN}
\subsection{Objetivo general}
Diseñar el prototipo de una aplicación móvil multiplataforma y de arquitectura distribuida para los sistemas operativos IOS y Android, que permita consultar menús saludables para niños en un rango de edad de 3 a 7 años, elaborado a partir un estudio que identifica cuáles son los factores que están generando malnutrición, causando "enfermedades del XXI " relacionadas con la alimentación. Para mejorar los hábitos alimenticios en la población infantil en este rango de edad. Se toma como muestra para la investigación el colegio Gimnasio Nuevo Despertar, ubicado en la localidad de Fontibón en Bogotá.\\ 
\subsection{Objetivos específicos}
\begin{itemize}
	\item Elaborar los instrumentos de investigación, que, al ser aplicados, permitan identificar cuáles son los factores que están generando mal nutrición por exceso en los niños con edad de 3 a 7 años en el colegio Gimnasio Nuevo Despertar, con el fin  de obtener los datos para el estudio.\\ 
	\item Identificar las causas que están generando malnutrición, basado en resultados obtenidos de la ejecución de los instrumentos de investigación tales como encuestas y entrevistas que al ser aplicados, puedan ser tomados como base base para la implementación del prototipo de la aplicación móvil propuesta.\\
	\item Diseñar un prototipo de aplicación móvil para sistemas operativos IOS y Android, que permita al usuario consultar menús de comida saludable para niños de 3 a 7 años, que, alineados a los resultados obtenidos en la investigación, permita mejorar los hábitos alimentación de este rango poblacional.	
\end{itemize}
\section{JUSTIFICACIÓN DE LA INVESTIGACIÓN}
\subsection{Justificación práctica}
Según datos de la Organización Mundial de la Salud (OMS), 41 millones de niños menores de cinco años tienen sobrepeso o son obesos en nuestro planeta [2] . La malnutrición ya sea por exceso o por desnutrición conlleva a problemas serios de salud en el ser humano, como el desarrollo de diversas enfermedades crónicas, tales como la diabetes y algunos tipos de cáncer.\\
Dado la magnitud del problema, a mediano y largo plazo esto deriva en importantes consecuencias económicas en el país de lo padece, porque limita la acumulación de capital humano y la capacidad de generar ingresos en la vida adulta.\\
En Colombia 1 de cada 4 niños en edad escolar sufre de sobre peso y uno de cada 5 es obeso, según cifras del Ministerio de Salud y Protección social [3].\\
Con este panorama, es necesario identificar cuales son las posibles causas de esta problemática  y plantear soluciones que permitan mejorar su estado actual, porque de otra manera en muy pocos años, esto va a derivar en serios problemas sociales y económicos para el país, con sobrecostos para el sistema de salud, dificultades de movilidad para los ciudadanos, aumento de numero de ciudadanos enfermos y discapacitados y por ende la disminución de la fuerza de trabajo y de la capacidad de crecimiento económico del país.
\section{HIPÓTESIS DE TRABAJO}
El desarrollo de una aplicación móvil es una forma eficiente y de gran impacto que ayuda a mejorar los hábitos alimenticios y la dieta de la población infantil de 3 a 7 años previniendo "Enfermedades del siglo XXI", dado el amplio alcance de las tecnologías de información y telecomunicaciones en la sociedad moderna y específicamente el uso de dispositivos móviles. 

